%%% Работа с русским языком
\usepackage[russian,english]{babel}   %% загружает пакет многоязыковой вёрстки
\usepackage{fontspec}      %% подготавливает загрузку шрифтов Open Type, True Type и др.
\defaultfontfeatures{Ligatures={TeX},Renderer=Basic,Scale=0.75}  %% свойства шрифтов по умолчанию
\setmainfont[Ligatures={TeX,Historic}]{Times New Roman} %% задаёт основной шрифт документа
\setsansfont[Ligatures={TeX,Historic}]{Verdana} %% задаёт шрифт без засечек
\setmonofont[Scale=0.7]{DejaVu Sans Mono}

\linespread{0.75}

%%% Работа с русским языком
\usepackage{cmap}					    % поиск в PDF
\usepackage{mathtext} 				% русские буквы в формулах

% Для разметки слайдов и картинок
\usepackage{tikz}
\usepackage{animate}

% Для двойных прямых кавычек в Verbatim (не работает почему-то)
\usepackage{upquote}

\definecolor{links}{HTML}{2A1B81}
\hypersetup{colorlinks=true,linkcolor=,urlcolor=links,pdfview=FitH,pdfpagelayout=SinglePage, unicode=true,breaklinks=true}

% links format
\usepackage{hyperref}

% decrease text margins
\setbeamersize{text margin left = 8pt, text margin right = 16pt}

\newcommand{\columnsbegin}{\begin{columns}}
\newcommand{\columnsend}{\end{columns}}
\newcommand{\blockbegin}{\begin{block}}
\newcommand{\blockend}{\end{block}}

\usepackage{graphics}

% allows to add alignment keys to \includegraphics
\usepackage[export]{adjustbox}

% % subfigures
% \usepackage{subcaption}
\usepackage{wrapfig}

%\addto{\captionsrussian}{\renewcommand*{\figurename}{рис.}}

% Tables
% Define new column types to adjust sizes in tabular environment
% For example, \begin{tabular}{ L{2.3cm} C{2cm} C{1.5cm} C{2.5cm} C{4cm}}
\usepackage{array}
\renewcommand{\arraystretch}{2}
\newcolumntype{L}[1]{>{\raggedright\let\newline\\\arraybackslash\hspace{0pt}}m{#1}}
\newcolumntype{C}[1]{>{\centering\let\newline\\\arraybackslash\hspace{0pt}}m{#1}}
\newcolumntype{R}[1]{>{\raggedleft\let\newline\\\arraybackslash\hspace{0pt}}m{#1}}

% Для таблиц в tabularx окружении при помощи пакета huxtable
\usepackage{tabularx}
\usepackage{multirow}
\usepackage{hhline} %обязательно
\usepackage{calc}

\logo{\includegraphics[height=0.3cm]{assets/Linmod_logo.png}}

% redefine beamer footline
\setbeamertemplate
{footline}{\quad\hfill\insertframenumber/\inserttotalframenumber\strut\quad}



